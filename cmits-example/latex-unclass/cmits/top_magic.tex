% CMITS - Configuration Management for Information Technology Systems
% Based on <https://github.com/afseo/cmits>.
% Copyright 2015 Jared Jennings <mailto:jjennings@fastmail.fm>.
%
% Licensed under the Apache License, Version 2.0 (the "License");
% you may not use this file except in compliance with the License.
% You may obtain a copy of the License at
%
%    http://www.apache.org/licenses/LICENSE-2.0
%
% Unless required by applicable law or agreed to in writing, software
% distributed under the License is distributed on an "AS IS" BASIS,
% WITHOUT WARRANTIES OR CONDITIONS OF ANY KIND, either express or implied.
% See the License for the specific language governing permissions and
% limitations under the License.

\usepackage{url}
\usepackage{cite}
\usepackage{verbatim}
\usepackage{color}

% Leave more room in the table of contents for section numbers, because in
% this document they get large (e.g., 12.10.151.)
\usepackage{tocloft}
\addtolength{\cftchapnumwidth}{0.3em}
\addtolength{\cftsecnumwidth}{0.8em}
\addtolength{\cftsubsecnumwidth}{1.4em}
\addtolength{\cftsubsubsecnumwidth}{1em}

% The tightverbatim environment is a verbatim environment with no
% paragraph spacing at the top and bottom. This is useful when we're
% popping in and out of verbatim all the time to set index entries, add
% labels and cross-references (via faxlore), to avoid essentially
% triple-spacing the Puppet source code.
%
% (The verbatim environment as written in
% /usr/share/texmf/tex/latex/tools/verbatim.sty skips space if we're not
% in a minipage. So this just makes us be in a minipage, so the space
% won't be skipped.)

\newenvironment{tightverbatim}%
    {\par\minipage{0.94\textwidth}\small\verbatim}%
    {\endverbatim\endminipage\par}

% http://ask.metafilter.com/31640/Whats-the-name-of-the-asterisks-that-divide-chapters
\newcommand{\dinkus}{%
    \begin{center}%
    *\hskip 3em *\hskip 3em *%
    \end{center}}

% Make PDF magic happen.
% Thanks, Patrick Joeckel:
% http://www.mpch-mainz.mpg.de/~joeckel/pdflatex/
% Note that I expect always to use pdflatex, not latex+dvipdf
\usepackage[pdftex,bookmarks=true,pdfpagelabels=true]{hyperref}
\makeatletter
\hypersetup{
    bookmarksnumbered=true,
    bookmarksopen=false,
    breaklinks=true,
    plainpages=false,
    colorlinks=false,
    linkbordercolor={0.7 0.7 1.0},
    pdftitle={\@title},
    pdfauthor={\@author}
}
\makeatother

\usepackage{cyber}
% must be used after hyperref - see source
\usepackage{cybercic}
\def\iamarginsize{\scriptsize}

\newindex{default}{default.idx}{default.ind}{Default Index}
\newindex{g}{general.idx}{general.ind}{General Index}
\newindex{class}{classes.idx}{classes.ind}{Classes}
\newindex{define}{defines.idx}{defines.ind}{Defined Resource Types}
\newindex{file}{files.idx}{files.ind}{Files}


% Make marginpars show up on the correct side of the page.
\usepackage{mparhack}

% Make the Bibliography a numbered chapter. In 1998, someone said you do
% this by redefining the \thebibliography environment, at
% http://groups.google.com/group/comp.text.tex/browse_thread/thread/b6e93d2f5e3a121a?hl=en
% This redefinition is based on the definition in
% /usr/share/texmf/tex/latex/base/book.cls from the
% tetex-latex-3.0-33.13.el5 package.

\makeatletter
\renewenvironment{thebibliography}[1]
     {\chapter{\bibname}% <<< star removed from this line
      \label{Bibliography}% <<< this line added
      \@mkboth{\MakeUppercase\bibname}{\MakeUppercase\bibname}%
      \list{\@biblabel{\@arabic\c@enumiv}}%
           {\settowidth\labelwidth{\@biblabel{#1}}%
            \leftmargin\labelwidth
            \advance\leftmargin\labelsep
            \@openbib@code
            \usecounter{enumiv}%
            \let\p@enumiv\@empty
            \renewcommand\theenumiv{\@arabic\c@enumiv}}%
      \sloppy
      \clubpenalty4000
      \@clubpenalty \clubpenalty
      \widowpenalty4000%
      \sfcode`\.\@m}
     {\def\@noitemerr
       {\@latex@warning{Empty `thebibliography' environment}}%
      \endlist}
\makeatother



