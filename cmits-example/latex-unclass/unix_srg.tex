% CMITS - Configuration Management for Information Technology Systems
% Based on <https://github.com/afseo/cmits>.
% Copyright 2015 Jared Jennings <mailto:jjennings@fastmail.fm>.
%
% Licensed under the Apache License, Version 2.0 (the "License");
% you may not use this file except in compliance with the License.
% You may obtain a copy of the License at
%
%    http://www.apache.org/licenses/LICENSE-2.0
%
% Unless required by applicable law or agreed to in writing, software
% distributed under the License is distributed on an "AS IS" BASIS,
% WITHOUT WARRANTIES OR CONDITIONS OF ANY KIND, either express or implied.
% See the License for the specific language governing permissions and
% limitations under the License.

\chapter{UNIX SRG Compliance}
\label{UNIXSRGCompliance}

This chapter has to do with the compliance of Linux machines controlled by
this policy, and administrators performing the procedures written here,
with the UNIX SRG \cite{unix-srg}.

In the indices of this document you can find a UNIX SRG Compliance Index.
All requirements directly and completely implemented by automated
application of this policy are listed in that index as ``implemented.''
The default Red Hat Enterprise Linux (RHEL) install satisfies some SRG
requirements; a list of those is below. In places where the SRG imposes
policy demands on the actions of administrators, those demands are passed
on in \S\ref{Procedures}. All other requirements are discussed in another
section below.

Where RHEL defaults to the correct behavior, but it is simple to write
automated policy that will fix anything that is broken, we do that, in an
attempt to ensure that UNIX hosts are not only compliant at rollout, but
remain compliant over time, and to ensure that noncompliance is rare
enough to draw attention where it is warranted.






\section{Requirements that RHEL implements by default}

\bydefault{RHEL5}{rhel5stig}{GEN000000-LNX007580}
\bydefault{RHEL5}{rhel5stig}{GEN000000-LNX007620} RHEL5 does not include LLC
support.

\bydefault{RHEL5, RHEL6}{unixsrg}{GEN000440} RHEL logs all logon attempts by default.

\bydefault{RHEL5, RHEL6}{unixsrg}{GEN000900} RHEL assigns the root user a home directory
of \verb!/root!, which is not \verb!/!.

\bydefault{RHEL5, RHEL6}{unixsrg}{GEN001040} RHEL logs all root logon attempts by
default.

\bydefault{RHEL5, RHEL6}{unixsrg}{GEN001060} RHEL logs all su attempts by default.

\bydefault{RHEL5, RHEL6}{unixsrg}{GEN001080} RHEL sets the root user's shell to
\verb!/bin/sh! by default.

\bydefault{RHEL5, RHEL6}{unixsrg}{GEN001220} \bydefault{RHEL5, RHEL6}{unixsrg}{GEN001240} RHEL ensures
by default that all system files, programs and directories are owned and
group-owned by system accounts, via its packaging system.

\bydefault{RHEL5, RHEL6}{iacontrol}{DCSL-1}
\bydefault{RHEL5, RHEL6}{unixsrg}{GEN001300} RHEL ensures by default that all system
library files have permissions of \verb!0755! or more restrictive, via its
packaging system.

\bydefault{RHEL5, RHEL6}{unixsrg}{GEN001100} RHEL comes with OpenSSH in the default
install, and telnet and rlogin/rsh not in the default install. A small
policy that provides defense in depth is in \S\ref{class_stig_misc}.

\bydefault{RHEL5, RHEL6}{unixsrg}{GEN001140} Neither RHEL5 nor RHEL6 as
shipped contain files with more permissions for group or other than for
user. But~\S\ref{class_stig_misc::find_uneven} checks for them anyway.

\bydefault{RHEL5, RHEL6}{unixsrg}{GEN001160}
\bydefault{RHEL5, RHEL6}{unixsrg}{GEN001170}
System files under RHEL are always owned by a valid user, because packages
that install those files add the corresponding user as necessary. By the
same token, packages under RHEL add any groups necessary to own system
files. But~\S\ref{class_stig_misc::find_unowned} checks for unowned files
anyway.

\bydefault{RHEL5, RHEL6}{unixsrg}{GEN001180}
\bydefault{RHEL5, RHEL6}{unixsrg}{GEN001190}
\bydefault{RHEL5, RHEL6}{unixsrg}{GEN001200}
\bydefault{RHEL5, RHEL6}{unixsrg}{GEN001210}
\bydefault{RHEL5, RHEL6}{unixsrg}{GEN001220}
\bydefault{RHEL5, RHEL6}{unixsrg}{GEN001240}
All system files, programs and directories under RHEL are owned and
group-owned by system users, and do not have extended ACLs. None have
write permission for any user but root, including executables relating to
network services.

With that said, see~\S\ref{class_stig_misc::permissions} for defense in depth.

\bydefault{RHEL5, RHEL6}{unixsrg}{GEN001300} All library files under RHEL are
mode \verb!0755! or less permissive by default.


\bydefault{RHEL5, RHEL6}{unixsrg}{GEN001660}
\bydefault{RHEL5, RHEL6}{unixsrg}{GEN001680} RHEL packages do not install any
non-root-owned system startup files.

\bydefault{RHEL5, RHEL6}{iacontrol}{DCSL-1}
\bydefault{RHEL5, RHEL6}{unixsrg}{GEN001700} RHEL packages do not install programs not
owned by a system account, so run control scripts cannot run such
programs.

\bydefault{RHEL5, RHEL6}{unixsrg}{GEN002240} We do not install any device files via
policy or procedure, so all device files are in the vendor-designated
directories as required.

\bydefault{RHEL5, RHEL6}{unixsrg}{GEN002280} RHEL makes \verb!/dev! writable only by the
owner, as required. As above, device files are only in \verb!/dev!.
World-writable device files are \verb!/dev/random!, \verb!/dev/urandom!,
\verb!ptmx!, \verb!/dev/null!, \verb!/dev/zero!, \verb!/dev/full!,
\verb!/dev/fuse!, \verb!/dev/net/tun!, \verb!/dev/tty!; these are all
world-writable by design. (Other STIG requirements have to do with
tunnelling; see the Unix SRG index for more on how we deal with them.)

\bydefault{RHEL5, RHEL6}{unixsrg}{GEN002320}
\bydefault{RHEL5, RHEL6}{unixsrg}{GEN002330}
\bydefault{RHEL5, RHEL6}{unixsrg}{GEN002340}
\bydefault{RHEL5, RHEL6}{unixsrg}{GEN002360}
Under RHEL default settings, console devices such as the floppy drive and
the microphone are managed by the pam\_console PAM module, which ensures
that the user who is logged in at the console owns these devices and no
one else can access them (mode \verb!0600!, no extended ACLs). This does
not comply with the letter of the requirement but does address the
vulnerability discussed therein.

\bydefault{RHEL5, RHEL6}{unixsrg}{GEN002640} System accounts are disabled by
default under RHEL.

\bydefault{RHEL5, RHEL6}{unixsrg}{GEN003540} Support for non-executable data
has been activated by default since RHEL3.

\bydefault{RHEL5, RHEL6}{unixsrg}{GEN003580} All Linux kernels since 1996
have improved TCP sequence number randomization, in material compliance
with this requirement.

\bydefault{RHEL5, RHEL6}{unixsrg}{GEN003640}
\bydefault{RHEL5, RHEL6}{unixsrg}{GEN003650} For the root filesystem and all
other local filesystems, RHEL5 and RHEL6 use the ext3 filesystem by
default, which is a journalling filesystem.

\bydefault{RHEL5, RHEL6}{unixsrg}{GEN003660} RHEL logs all successful and
unsuccessful logins by default.

\bydefault{RHEL5, RHEL6}{unixsrg}{GEN005060} Neither RHEL5 nor RHEL6 provides
an FSP server, nor do we deploy one.

\bydefault{RHEL5, RHEL6}{unixsrg}{GEN005160} 
\bydefault{RHEL5, RHEL6}{unixsrg}{GEN005180}
\bydefault{RHEL5, RHEL6}{unixsrg}{GEN005190}
\bydefault{RHEL5, RHEL6}{unixsrg}{GEN005220}
RHEL X servers write \verb!.Xauthority! files by default, with mode
\verb!0600! and no extended ACLs, as required, and use them for access
restriction.

\bydefault{RHEL5, RHEL6}{unixsrg}{GEN005240}
\bydefault{RHEL5, RHEL6}{unixsrg}{GEN005260} RHEL X servers do not listen for
network connections by default, so users cannot permit X display access to
unauthorized hosts.

\bydefault{RHEL5, RHEL6}{unixsrg}{GEN006000} RHEL does not provide AOL
Instant Messenger (AIM), MSN Messenger, or Yahoo! Messenger. It does
provide the Pidgin instant messaging client, which is the means by which
users connect to the DISA-sponsored Defense Connect Online (DCO) instant
messaging service from RHEL. According to the discussion of this
requirement, ``Clients used to access internal or DoD-controlled IM
services are permitted.''

\bydefault{RHEL5, RHEL6}{unixsrg}{GEN006040} RHEL does not provide any
peer-to-peer file sharing applications.

\bydefault{RHEL6}{unixsrg}{GEN006240} RHEL6 does not provide any Usenet news
server software.

\bydefault{RHEL5,RHEL6}{unixsrg}{GEN007960} Upon inspection of the source
code of the \verb!ldd! command both under RHEL5 and RHEL6, it does not run
the executable in question, but hands it as a parameter to the dynamic
linker. This means that according to the vulnerability discussion, the
\verb!ldd! command suitably ``protects against the execution of untrusted
files.''

\bydefault{RHEL5,RHEL6}{unixsrg}{GEN008420} RHEL has had the Exec Shield
technology for address randomization since RHEL3 update 3. See
\url{http://people.redhat.com/drepper/nonselsec.pdf} and
\url{http://www.redhat.com/f/pdf/rhel/WHP0006US_Execshield.pdf}.

\bydefault{RHEL5, RHEL6}{unixsrg}{GEN002480}
\bydefault{RHEL5, RHEL6}{unixsrg}{GEN002500}
\bydefault{RHEL5, RHEL6}{unixsrg}{GEN002520}
\bydefault{RHEL5, RHEL6}{unixsrg}{GEN002540}
RHEL public directories are as follows. All public directories are owned
by root, group root, and have the sticky bit set. No other world-writable
directories exist on a stock RHEL system.
\begin{itemize}
\item \verb!/tmp!
\item \verb!/tmp/.ICE-unix!
\item \verb!/tmp/.X11-unix!
\item \verb!/tmp/.font-unix!
\item \verb!/var/tmp!
\item \verb!/usr/src/debug/tmp!
\end{itemize}

When installed, VMware Workstation installs a public directory for
drag-and-drop functionality, \verb!/tmp/VMwareDnD!. It also fulfills the
SRG requirements.

% CMITS - Configuration Management for Information Technology Systems
% Based on <https://github.com/afseo/cmits>.
% Copyright 2015 Jared Jennings <mailto:jjennings@fastmail.fm>.
%
% Licensed under the Apache License, Version 2.0 (the "License");
% you may not use this file except in compliance with the License.
% You may obtain a copy of the License at
%
%    http://www.apache.org/licenses/LICENSE-2.0
%
% Unless required by applicable law or agreed to in writing, software
% distributed under the License is distributed on an "AS IS" BASIS,
% WITHOUT WARRANTIES OR CONDITIONS OF ANY KIND, either express or implied.
% See the License for the specific language governing permissions and
% limitations under the License.
% included by unix_srg

\section{Requirements that are not applicable}

% CMITS - Configuration Management for Information Technology Systems
% Based on <https://github.com/afseo/cmits>.
% Copyright 2015 Jared Jennings <mailto:jjennings@fastmail.fm>.
%
% Licensed under the Apache License, Version 2.0 (the "License");
% you may not use this file except in compliance with the License.
% You may obtain a copy of the License at
%
%    http://www.apache.org/licenses/LICENSE-2.0
%
% Unless required by applicable law or agreed to in writing, software
% distributed under the License is distributed on an "AS IS" BASIS,
% WITHOUT WARRANTIES OR CONDITIONS OF ANY KIND, either express or implied.
% See the License for the specific language governing permissions and
% limitations under the License.
% included by unix_srg chapter

\section{Requirements we may not be meeting}

% CMITS - Configuration Management for Information Technology Systems
% Based on <https://github.com/afseo/cmits>.
% Copyright 2015 Jared Jennings <mailto:jjennings@fastmail.fm>.
%
% Licensed under the Apache License, Version 2.0 (the "License");
% you may not use this file except in compliance with the License.
% You may obtain a copy of the License at
%
%    http://www.apache.org/licenses/LICENSE-2.0
%
% Unless required by applicable law or agreed to in writing, software
% distributed under the License is distributed on an "AS IS" BASIS,
% WITHOUT WARRANTIES OR CONDITIONS OF ANY KIND, either express or implied.
% See the License for the specific language governing permissions and
% limitations under the License.
% This file is included by the unix_srg chapter.
%
% This file should be superseded by a more organization-specific file, so
% it'll probably only show up as it is below in the example policy.

\section{Things required to be documented with the IAO}
\label{DocumentedWithIAO}

Several SRG requirements say that this or that thing must be ``documented
with the IAO'' (Information Assurance Officer). This section should point
readers to the places where that documentation resides, or in degenerate
cases (``We don't have any of these things that must be documented with
the IAO'') just say so.


